% \blindtext

% the problem
\gls{nft}s allow people to trace the origin of digital items and with the help of Blockchain technology. Since the items are unique from each other, as expressed by the name itself, they are \textit{not fungible}. One \gls{nft} is expected to be unique from another. 
Due to several restraints that are presented with the nature of \gls{nft}s \& the overwhelming amount of data that needs to be analyzed, it is difficult to find NFTs of comparable value that is trending among the community, timely and relevant to each user’s identified interests or the NFT that the user currently owns.

% problem solving methodology
Recommendations Systems have been identified to be one of the integral elements of driving sales in e-commerce cites. The utilization of opinion mining data extracted from trends have been attempted to improve the recommendations that can be provided by baseline methods in this research, to address the restraints presented by \gls{nft}s.

% discuss the result
\textit{\gls{nft}-RecSys} is capable of acting as a decentralized Recommendations System to provide trending recommendations of \gls{nft} assets, while preserving user-anonymity. The data extraction methods explored for recommending \gls{nft}s, integration of social-trends into recommendations \& the aggregation algorithm of recommendations from ensembled models are novel results yielded by this research.
% add more results based on testing & evaluation

\bigbreak
\textbf{Keywords:} Recommendation Systems, Non-fungible Tokens, Data Science, Opinion Mining, Hybrid Recommendation Systems, Machine Learning, Decentralized Systems