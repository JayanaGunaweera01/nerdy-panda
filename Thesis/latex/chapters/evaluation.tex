% Hasal's commented thesis for structure: https://drive.google.com/drive/u/1/folders/1kYS3apmM6mKPFHErfuw4VTdnDHqAMbfW

\section{Chapter Overview}
After the designed prototype had been successfully implemented and was optimized to achieve the best performance through a large number of testing combinations, the system was evaluated with respective to the requirements gathered in the SRS chapter. This chapter is dedicated to the project's evaluation, which will involve self-evaluation as well as assessments from technical, domain and industry experts.

\section{Evaluation Methodology \& Approach}
Since the research project consists of models that can be quantitatively evaluated and another model that presents a more qualitative output, both qualitative \& quantitative evaluation approaches were taken. Based on the tests carried out in the testing chapter, the research outcome given by the prototype was evaluated using evaluation techniques of Recommendation Systems extracted from literature. In this chapter, a thematic analysis will be used to present the feedback received from experts.

% Upload video to Youtube as unlisted video and provide Link to video demo
\noindent The link to the demonstration video of the research, that was used for evaluations can be found here: \url{https://youtu.be/fjRzZXUOrRo}

\section{Evaluation Criteria}

The following criteria were used for the thematic analysis that surfaced in interviews with experts \& other aspects of research that needed to be assessed to determine the value of the research that was conducted.

\vspace{-4mm}
\begin{longtable}{|p{0.22\linewidth}|p{0.726\linewidth}|}
\caption{Evaluation Criteria}\\ 
\hline
\textbf{Criterion} & \textbf{Evaluation Purpose} \endfirsthead 
\hline
Choice of research undertaken & To validate the significance of the choices of topic, domain, research gap, and depth undertaken in this research. \\ 
\hline
Research contribution & To determine the value of the contributions produced to the technical field of Recommendation Systems, the domain of \gls{nft}s or Blockchain and any other additional research-oriented contributions made. \\ 
\hline
Quality of research documentation & To confirm that an adequate amount of literature has been reviewed and the entire research process has been documented \& presented in an acceptable quality. \\ 
\hline
Development approach & To confirm that an appropriate development approach had been taken to solve the problem at hand to the best possible extent, with the implementation of the prototype.\\ 
\hline
Quantitative analysis of results & To validate the metrics used to evaluate \& analyze the results produced by the research. \\ 
\hline
Possible improvements & To unveil possible improvements that could be worked on as Future Work related to the conducted research. \\
\hline
Usability, UI/ UX of MVP &
To verify that the product developed for demonstration is convenient for end-users. \\
\hline
\end{longtable}

\section{Self-Evaluation}

The following self-evaluation was done by the author of the research according to the above-mentioned evaluation criteria.

\vspace{-2mm}
\begin{longtable}{|p{0.22\linewidth}|p{0.726\linewidth}|}
\caption{Self-evaluation of the author according to the Evaluation Criteria}\\ 
\hline
\textbf{Criterion} & \textbf{Author's Self-evaluation} \endfirsthead 
\hline
Choice of research undertaken & The research area chosen revolved around a highly useful technical application as well as a very new \& popular domain that is expected to be used in many applications in the future. \\ 
\hline
Research contribution & The contributions of this research lie across a broad spectrum. Firstly, the technical contributions made in Recommendation Systems can be identified by the novel recommendations method introduced with the use of a custom algorithm to recommend trending items based on social media trends. Secondly, the contribution to the domain is novel \& has opened new pathways to possible future work. \\ 
\hline
Quality of research documentation & The quality of the documentation is of the highest possible standard. The use of Latex for all the research documentation including the thesis signifies this, together with the quality of the diagrams \& content as well as the research papers written. \\
\hline
Development approach & A significant effort has been put into data collection \& pre-processing to give the best possible results with a very meager amount of data. Cutting-edge languages \& tools have also been used in the process.

In the requirements of a Data-scientist job at OpenSea, which was posted on LinkedIn shown in \textit{\nameref{appendix:evaluation} }emphasizes the value and requirement of the approach taken to recommend items using social media trends \& sentiment.\\
\hline
Quantitative analysis of results & Even though the quantitative analysis \& evaluation of the results produced by the system is difficult to be measured, Jupyter notebooks have been used to demonstrate these using graphical outputs in a comprehensible format. \\ 
\hline
Possible improvements & After getting expert \& domain evaluators' feedback, possible improvements that could be addressed were attempted. Especially a more comprehensive quantitative evaluation of the trends-based model. \\
\hline
Usability, UI/ UX of MVP & The UI/UX of the final product has been developed in a very usable \& attractive manner. \\
\hline
\end{longtable}


\section{Selection of Evaluators}
The selection categories of evaluators for the project can be broken down into the following 3 categories.
% Evaluations were received by Blockchain domain experts, Data Science \& Engineering experts \& possible end-users of the application.

\vspace{-4mm}
\begin{longtable}{|p{0.05\linewidth}|p{0.886\linewidth}|}
\caption{Categorization of selected evaluators}
\label{tab:categories-of-evaluators}
\\ 
\hline
\textbf{CAT ID} & \textbf{Category} \endfirsthead 
\hline
1 & Experts with research experience in the fields of Recommendation Systems, Data Science, Data Engineering \& Machine Learning. \\ 
\hline
2 & Experts with domain expertise in the fields of Blockchain, DApp (Decentralized App) Development \& \gls{nft}s. \\ 
\hline
3 & Possible end-users of the applications such as \gls{nft} creators, collectors \& enthusiasts. \\ 
\hline
\end{longtable}

\section{Evaluation Results \& Expert Opinions}
% have a table with all the feedback received by each evaluator in the appendix (same table - % have their positions/ relevance to the project).

% Feedback that was received in the evaluation process by each evaluator has been mentioned in \nameref{appendix}.

\subsection{Qualitative Analysis}
% form responses sheet: https://docs.google.com/spreadsheets/d/172tcoWh9IjTteXMYUhxZTp-gBK6caxrhLB-aQhq7-mc/edit?resourcekey#gid=1740062709
% create in google docs -> 2 sections of this table in tables editor & 
% \vspace{-2mm}
The expert opinions that were received have been analyzed according to emerged themes below.
% as feedback when demonstrating the prototype after explaining the research gap 

% \vspace{-4mm}
\begin{longtable}{|p{0.16\linewidth}|p{0.05\linewidth}|p{0.22\linewidth}|p{0.46\linewidth}|}
\caption{Thematic analysis of expert evaluation feedback}\\ 
\hline
\textbf{Criterion} & \textbf{CAT ID} & \textbf{Theme} & \textbf{Summary of Opinions} \\* 
\hline
\multirow{4}{*}{\parbox{\linewidth}{Choice of research undertaken}} & \multirow{2}{*}{1} & Recommendation Systems choice  gap & The study of Recommendation Systems is valuable  impactful, especially in an unexplored e-commerce domain that is new and has had high trading volumes. \\* 
\cline{3-4}
 &  & Technical research gap & Due to the nature of NFTs, it was good not to stick to the standard collaborative filtering method. This opened up a good research area. \\* 
\cline{2-4}
 & 2 & Domain research gap & The domain is new. There’s a clear research gap identified to be fulfilled. \\* 
\cline{2-4}
 & 3 & Domain research applicability for use & The domain application is new \& required since it’s difficult to explore items. Nothing like what the author has attempted in his research has been attempted before. It’s interesting. \\* 
\hline
\multirow{3}{*}{\parbox{\linewidth}{Research contribution}} & 1 & Technical Contribution towards Recommendation Systems & Innovative methods of solving the research gap have been identified. The clear gap \& issue with traditional Recommendation methods have been addressed. \\* 
\cline{2-4}
 & 2 & Domain Contribution & The contribution is good because there’s no other system like this and gathering data is difficult. \\* 
\cline{2-4}
 & 3 & Domain Contribution & It’s a major contribution towards the domain. The author has been successful at unveiling impactful recommendation techniques \& features considered. \\* 
\hline
\multirow{2}{*}{\parbox{\linewidth}{Quality of research documentation}} & 1 & Content  presentation of content & The use of latex was immediately noticed and commended. In-depth research has been conducted with the presentation of statistics.~ \\* 
\cline{2-4}
 & 1, 2, 3 & Approach Taken to achieve the solution to the problem & Many angles have been considered to approach the solution. A scientific approach has been taken. \\* 
\hline
\multirow{3}{*}{\parbox{\linewidth}{Development approach}} & 1 & Data preprocessing & A great amount of data pre-processing has been done, as it should be for a Recommendations System to produce optimum results. \\* 
\cline{2-4}
 & 1, 2 & Selection of the Sentiment Analysis model & The selection  use of the output of the Sentiment Analysis model has been well-thought-out and justified. \\* 
\cline{2-4}
 & 3 & Development approach & Everyone agreed that it was a good direction to take and a straightforward and methodical approach has been taken, without jumping directly to model construction. \\* 
\hline
\multirow{2}{*}{\parbox{\linewidth}{Quantitative analysis of results}} & 1, 2 & Analysis of the social trends based RecSys & The current evaluation method makes sense and is clearly understandable. Could try to scrape the internet/ OpenSea to validate generated recommendations from the social trends RecSys. Could try to synthesize  demonstrate results from the social trends RecSys to show why it’s needed. Technical aspects have been well-evaluated. it would be good if the researcher get feedback from people based on the recommendations produced. Model accuracy could’ve been emphasized. \\* 
\cline{2-4}
 & 1, 2, 3 & Analysis of trait-based RecSys & The graphical analysis of the models is very clear. Looks like the best way to evaluate these models. \\* 
\hline
\multirow{3}{*}{\parbox{\linewidth}{Possible improvements}} & 1 & Additionally considerable parameters & Consider the price for recommendations. \\* 
\cline{2-4}
 & 2 & Analysis of the social trends based RecSys & Could have been evaluated by different parameters such as country, age, domain(art, game NFT), etc.~ Consider NFT utilities, buy and sell trend, no of bids, and visibility as well to identify the top NFTs. \\* 
\cline{2-4}
 & 3 & Credibility of source & Identify artists by tracking their activity to recommend items by credible artists. Identifying fraudulent NFTs would also be interesting. \\* 
\hline
\multirow{2}{*}{\parbox{\linewidth}{Usability, UI/ UX of MVP}} & 1 & Requirement of a separate application & Since the prototype produces clear results, a separate application is not required. \\* 
\cline{2-4}
 & 3 & Present descriptions of recommendations & Would’ve been able to better understand the results in comparison with the reference NFT if the details of them were displayed. \\
\hline
\end{longtable}


% \subsection{Quantitative Analysis}
% should I take Hasal's approach where he discusses his responses or evaluate the quantitative results of the testing chapter?
% this seems useless for my research

\section{Limitations of Evaluation}
As discussed in the literature review, it is very difficult to evaluate a Recommendations System, especially one that is specific to a particular use case. Therefore, the testing \& evaluation equations had to be adjusted to suit these.
The social trends-based Recommendations System was the most difficult to evaluate due to the lack of data that was available to the author.
Since the domain is very new, there were very few people who could understand the impact of the domain contribution \& why some choices had to be made.\\
Long hours of power-cuts throughout the evaluation phase of the project made it very difficult to set up meetings with evaluators and work on evaluation aspects of the research.

\section{Evaluation of Functional Requirements}
The breakdown of the evaluation of functional requirements can be found in the table \nameref{tab:eval-func-requirements} of \textbf{\nameref{appendix:evaluation}}

\section{Evaluation of Non-functional Requirements}
The breakdown of the evaluation of non-functional requirements can be found in the table \nameref{tab:eval-non-func-requirements} of \textbf{\nameref{appendix:evaluation}}

\section{Chapter Summary}
This chapter covered the evaluation aspects of the research that was conducted. The approaches taken for evaluation were discussed with the reasoning of choosing each method. The criteria for evaluation were defined prior to the author's self-evaluation \& feedback from evaluators. The opinions received from evaluators were broken down into themes and presented based on the pre-defined criterion. Finally, the functional \& non-functional requirements were evaluated.